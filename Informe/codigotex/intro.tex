\section*{Resumen}
\noindent\textbf{Palabras claves:} Método de Elementos de Borde, BEM, BEM++, Micro-hilos, Sensor,\\ Dispersión Múltiple.
%%%%%%%%%%%%%%%%%%%%%%%%%%%%%%%%%%%%%%%%%%%%%%%%%%%%%%%%%%%
\pagebreak
\section*{Abstract}

\noindent\textbf{Keywords:} Boundary Element Method, BEM, BEM++, Micro-wires, Sensor,\\ Multiple Scattering.
%%%%%%%%%%%%%%%%%%%%%%%%%%%%%%%%%%%%%%%%%%%%%%%%%%%%%%%%%%%
\pagebreak
\section*{Glosario}
\begin{itemize}
	\renewcommand\labelitemi{--}
	\item \textbf{Dispersión electromagnética:} proceso en que una onda electromagnética incidente se desvía de su trayectoria a causa de pasar por una no uniformidad. Principalmente se distinguen los procesos de reflexión y refracción. 
	\item \textbf{Número de onda:} se puede definir como el número de ondas que existen en una determinada distancia. Frecuencia espacial.
	\item \textbf{Índice de refracción:} número adimensional que describe cómo se propaga una onda electromagnética en un medio.
	\item \textbf{Material dieléctrico:} material que exhibe polarización frente un campo eléctrico. 
	\item \textbf{Material conductor:} material que permite fluir sus electrones frente a un campo eléctrico. La capacidad de fluir libremente depende de la resistencia y conductividad del material.
	\item \textbf{Microondas:} forma de radiación electromagnética con longitudes de onda desde 1[m] a 1[mm], o con frecuencias desde los 300[MHz] hasta los 30[GHz].
	\item \textbf{Permitividad:} medida de resistencia encontrada cuando se forma un campo eléctrico en un medio.
	\item \textbf{Permeabilidad (magnética):} medida de la capacidad de un material para formar un campo magnético en él.
	\item \textbf{Permitividad relativa:} representación de la permitividad de un medio como un cociente de la permitividad absoluta y la permitividad del vacío. 
	\item \textbf{Permeabilidad relativa:} representación de la permeabilidad de un medio como un cociente de la permeabilidad absoluta y la permeabilidad del vacío. 
	\item \textbf{Compósito:} material compuesto por dos o más materiales con propiedades muy diferentes, que cuando se encuentran combinados presentan características distintas a las de los materiales individuales.  
	\item \textbf{SEM:} un microscopio de escáner por electrones (scanning electron microscope) es una clase de microscopio que produce imágenes de una muestra a través de escaneo por un haz enfocado de electrones.
	\item \textbf{Polarización:} En ondas es la habilidad de oscilar en más de una dirección. Para el caso de materiales (dieléctricos) significa la redistribución de electrones al interior de este producto de un campo eléctrico.
	\item \textbf{Impedancia:} es el cociente complejo entre el voltaje y la corriente generada. Caso similar a la resistencia eléctrica con la diferencia que la resistencia sólo tiene magnitud y la impedancia magnitud y fase.
	\item \textbf{Magneto impedancia:} cambio de la impedancia producto de un campo magnético.
	\item \textbf{Giant Magnetoimpedance (GMI):} corresponde a grandes variaciones en la impedancia eléctrica exhibidas por algunos materiales en función de un campo magnético externo.
	\item \textbf{Anisotropía magnética:} dependencia direccional de las propiedades magnéticas de un material.
	\item \textbf{Impedancia de superficie:} corresponde a la impedancia presentada por un conductor en su superficie como aproximación a su funcionamiento interno. 
	\item \textbf{Magneto estricción:} propiedad de los materiales ferromagnéticos que produce su cambio de forma o dimensiones durante el proceso de magnetización. Estos cambios serán producidos producto de un campo magnético hasta alcanzar su punto de saturación.
	\item \textbf{Análisis numérico:} estudio de algoritmos para aproximar problemas matemáticos complejos.
	\item \textbf{Boundary Element Method (BEM):} método numérico para aproximar ecuaciones diferenciales por medio de una formulación integral en el borde del dominio.
	\item \textbf{Python:} lenguaje de programación de alto nivel para uso variado.
	\item \textbf{BEM++:} librería abierta para Python que tiene como finalidad resolver problemas a través del método de elementos de borde.	
\end{itemize}
\chapter{Introducción.}
\pagestyle{plain}
\setcounter{page}{1}
\pagenumbering{arabic}
Durante los últimos años el desarrollo de nuevos métodos de resolución de ecuaciones diferenciales ha llamado la atención de la comunidad cientifica, siendo el método de elementos finitos el más conocido, utilizado y estudiado; este método realiza una discretización volumétrica completa del cuerpo de estudio, por lo que es muy útil para geometrías complejas. Sin embargo, este método no es el más eficiente en todos los casos.\\
Dentro de la gama de métodos con los que es posible resolver ecuaciones diferenciales parciales, el método de elementos de borde (conocido también como método de elementos de contorno o de frontera), BEM por sus siglas en inglés, es un método particularmente eficiente cuando tratamos con algo de homogeneidad dentro del cuerpo de estudio, esto usualmente sucede en mecánica de fluidos, acústica, electromagnetismo y algunos problemas de mecánica.\\
Si comparamos ambos métodos vemos una considerable cantidad de software que utilizan el método de elementos finitos para la resolución de problemas, sin embargo, BEM está aún 'en pañales', en etapa de desarrollo, por lo que los avances en la implementación del método, mejoras en la velocidad de resolución, mayor presición a menor costo computacional, entre otros factores, son objetivos a lograr dentro del corto y mediano plazo.\\
%\pagenumbering{arabic}